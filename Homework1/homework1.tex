% Options for packages loaded elsewhere
\PassOptionsToPackage{unicode}{hyperref}
\PassOptionsToPackage{hyphens}{url}
\PassOptionsToPackage{dvipsnames,svgnames,x11names}{xcolor}
\documentclass[
  11pt,
]{article}
\usepackage{xcolor}
\usepackage[margin=0.5in]{geometry}
\usepackage{amsmath,amssymb}
\setcounter{secnumdepth}{5}
\usepackage{iftex}
\ifPDFTeX
  \usepackage[T1]{fontenc}
  \usepackage[utf8]{inputenc}
  \usepackage{textcomp} % provide euro and other symbols
\else % if luatex or xetex
  \usepackage{unicode-math} % this also loads fontspec
  \defaultfontfeatures{Scale=MatchLowercase}
  \defaultfontfeatures[\rmfamily]{Ligatures=TeX,Scale=1}
\fi
\usepackage{lmodern}
\ifPDFTeX\else
  % xetex/luatex font selection
  \setmainfont[]{Helvetica}
  \setmonofont[]{Menlo}
\fi
% Use upquote if available, for straight quotes in verbatim environments
\IfFileExists{upquote.sty}{\usepackage{upquote}}{}
\IfFileExists{microtype.sty}{% use microtype if available
  \usepackage[]{microtype}
  \UseMicrotypeSet[protrusion]{basicmath} % disable protrusion for tt fonts
}{}
\makeatletter
\@ifundefined{KOMAClassName}{% if non-KOMA class
  \IfFileExists{parskip.sty}{%
    \usepackage{parskip}
  }{% else
    \setlength{\parindent}{0pt}
    \setlength{\parskip}{6pt plus 2pt minus 1pt}}
}{% if KOMA class
  \KOMAoptions{parskip=half}}
\makeatother
\usepackage{color}
\usepackage{fancyvrb}
\newcommand{\VerbBar}{|}
\newcommand{\VERB}{\Verb[commandchars=\\\{\}]}
\DefineVerbatimEnvironment{Highlighting}{Verbatim}{commandchars=\\\{\}}
% Add ',fontsize=\small' for more characters per line
\usepackage{framed}
\definecolor{shadecolor}{RGB}{248,248,248}
\newenvironment{Shaded}{\begin{snugshade}}{\end{snugshade}}
\newcommand{\AlertTok}[1]{\textcolor[rgb]{0.94,0.16,0.16}{#1}}
\newcommand{\AnnotationTok}[1]{\textcolor[rgb]{0.56,0.35,0.01}{\textbf{\textit{#1}}}}
\newcommand{\AttributeTok}[1]{\textcolor[rgb]{0.13,0.29,0.53}{#1}}
\newcommand{\BaseNTok}[1]{\textcolor[rgb]{0.00,0.00,0.81}{#1}}
\newcommand{\BuiltInTok}[1]{#1}
\newcommand{\CharTok}[1]{\textcolor[rgb]{0.31,0.60,0.02}{#1}}
\newcommand{\CommentTok}[1]{\textcolor[rgb]{0.56,0.35,0.01}{\textit{#1}}}
\newcommand{\CommentVarTok}[1]{\textcolor[rgb]{0.56,0.35,0.01}{\textbf{\textit{#1}}}}
\newcommand{\ConstantTok}[1]{\textcolor[rgb]{0.56,0.35,0.01}{#1}}
\newcommand{\ControlFlowTok}[1]{\textcolor[rgb]{0.13,0.29,0.53}{\textbf{#1}}}
\newcommand{\DataTypeTok}[1]{\textcolor[rgb]{0.13,0.29,0.53}{#1}}
\newcommand{\DecValTok}[1]{\textcolor[rgb]{0.00,0.00,0.81}{#1}}
\newcommand{\DocumentationTok}[1]{\textcolor[rgb]{0.56,0.35,0.01}{\textbf{\textit{#1}}}}
\newcommand{\ErrorTok}[1]{\textcolor[rgb]{0.64,0.00,0.00}{\textbf{#1}}}
\newcommand{\ExtensionTok}[1]{#1}
\newcommand{\FloatTok}[1]{\textcolor[rgb]{0.00,0.00,0.81}{#1}}
\newcommand{\FunctionTok}[1]{\textcolor[rgb]{0.13,0.29,0.53}{\textbf{#1}}}
\newcommand{\ImportTok}[1]{#1}
\newcommand{\InformationTok}[1]{\textcolor[rgb]{0.56,0.35,0.01}{\textbf{\textit{#1}}}}
\newcommand{\KeywordTok}[1]{\textcolor[rgb]{0.13,0.29,0.53}{\textbf{#1}}}
\newcommand{\NormalTok}[1]{#1}
\newcommand{\OperatorTok}[1]{\textcolor[rgb]{0.81,0.36,0.00}{\textbf{#1}}}
\newcommand{\OtherTok}[1]{\textcolor[rgb]{0.56,0.35,0.01}{#1}}
\newcommand{\PreprocessorTok}[1]{\textcolor[rgb]{0.56,0.35,0.01}{\textit{#1}}}
\newcommand{\RegionMarkerTok}[1]{#1}
\newcommand{\SpecialCharTok}[1]{\textcolor[rgb]{0.81,0.36,0.00}{\textbf{#1}}}
\newcommand{\SpecialStringTok}[1]{\textcolor[rgb]{0.31,0.60,0.02}{#1}}
\newcommand{\StringTok}[1]{\textcolor[rgb]{0.31,0.60,0.02}{#1}}
\newcommand{\VariableTok}[1]{\textcolor[rgb]{0.00,0.00,0.00}{#1}}
\newcommand{\VerbatimStringTok}[1]{\textcolor[rgb]{0.31,0.60,0.02}{#1}}
\newcommand{\WarningTok}[1]{\textcolor[rgb]{0.56,0.35,0.01}{\textbf{\textit{#1}}}}
\usepackage{graphicx}
\makeatletter
\newsavebox\pandoc@box
\newcommand*\pandocbounded[1]{% scales image to fit in text height/width
  \sbox\pandoc@box{#1}%
  \Gscale@div\@tempa{\textheight}{\dimexpr\ht\pandoc@box+\dp\pandoc@box\relax}%
  \Gscale@div\@tempb{\linewidth}{\wd\pandoc@box}%
  \ifdim\@tempb\p@<\@tempa\p@\let\@tempa\@tempb\fi% select the smaller of both
  \ifdim\@tempa\p@<\p@\scalebox{\@tempa}{\usebox\pandoc@box}%
  \else\usebox{\pandoc@box}%
  \fi%
}
% Set default figure placement to htbp
\def\fps@figure{htbp}
\makeatother
\setlength{\emergencystretch}{3em} % prevent overfull lines
\providecommand{\tightlist}{%
  \setlength{\itemsep}{0pt}\setlength{\parskip}{0pt}}
\usepackage{xcolor}
\usepackage{fvextra}
\DefineVerbatimEnvironment{Highlighting}{Verbatim}{breaklines,breakanywhere,commandchars=\\\{\}}
\usepackage{graphicx}
\usepackage{float}
\usepackage{microtype}
\usepackage{setspace}
\setstretch{1.05}
\setlength{\emergencystretch}{1em}
\makeatletter
\def\maxwidth{\ifdim\Gin@nat@width>\linewidth\linewidth\else\Gin@nat@width\fi}
\def\maxheight{\ifdim\Gin@nat@height>0.85\textheight0.85\textheight\else\Gin@nat@height\fi}
\makeatother
\setkeys{Gin}{width=\maxwidth,height=\maxheight,keepaspectratio}
\floatplacement{figure}{H}
\usepackage{bookmark}
\IfFileExists{xurl.sty}{\usepackage{xurl}}{} % add URL line breaks if available
\urlstyle{same}
\hypersetup{
  pdftitle={FPP3 Chapter 2 Graphics Exercise},
  pdfauthor={Randy Howk},
  colorlinks=true,
  linkcolor={blue},
  filecolor={Maroon},
  citecolor={Blue},
  urlcolor={blue},
  pdfcreator={LaTeX via pandoc}}

\title{FPP3 Chapter 2 Graphics Exercise}
\author{Randy Howk}
\date{February 08, 2026}

\begin{document}
\maketitle

{
\setcounter{tocdepth}{2}
\tableofcontents
}
\section{Explore four time series}\label{explore-four-time-series}

Series explored:

\begin{itemize}
\tightlist
\item
  \textbf{Bricks} from \texttt{aus\_production}
\item
  \textbf{Lynx} from \texttt{pelt}
\item
  \textbf{Close} from \texttt{gafa\_stock}
\item
  \textbf{Demand} from \texttt{vic\_elec}
\end{itemize}

\subsection{Dataset help pages}\label{dataset-help-pages}

\begin{Shaded}
\begin{Highlighting}[]
\NormalTok{?aus\_production}
\NormalTok{?pelt}
\NormalTok{?gafa\_stock}
\NormalTok{?vic\_elec}
\end{Highlighting}
\end{Shaded}

\subsection{Time interval of each
series}\label{time-interval-of-each-series}

Below, I compute the interval from each tsibble index.

\begin{Shaded}
\begin{Highlighting}[]
\NormalTok{bricks\_ts }\OtherTok{\textless{}{-}}\NormalTok{ aus\_production }\SpecialCharTok{|\textgreater{}} \FunctionTok{select}\NormalTok{(Quarter, Bricks)}
\NormalTok{lynx\_ts }\OtherTok{\textless{}{-}} \FunctionTok{pelt\_long}\NormalTok{(pelt) }\SpecialCharTok{|\textgreater{}} \FunctionTok{filter}\NormalTok{(Animal }\SpecialCharTok{==} \StringTok{"Lynx"}\NormalTok{)}
\NormalTok{close\_ts  }\OtherTok{\textless{}{-}}\NormalTok{ gafa\_stock }\SpecialCharTok{|\textgreater{}} \FunctionTok{select}\NormalTok{(Symbol, Date, Close)}
\NormalTok{demand\_ts }\OtherTok{\textless{}{-}}\NormalTok{ vic\_elec }\SpecialCharTok{|\textgreater{}} \FunctionTok{select}\NormalTok{(Time, Demand)}

\NormalTok{bricks\_interval }\OtherTok{\textless{}{-}}\NormalTok{ bricks\_ts }\SpecialCharTok{|\textgreater{}}\NormalTok{ tsibble}\SpecialCharTok{::}\FunctionTok{interval}\NormalTok{()}
\NormalTok{lynx\_interval   }\OtherTok{\textless{}{-}}\NormalTok{ lynx\_ts   }\SpecialCharTok{|\textgreater{}}\NormalTok{ tsibble}\SpecialCharTok{::}\FunctionTok{interval}\NormalTok{()}
\NormalTok{close\_interval  }\OtherTok{\textless{}{-}}\NormalTok{ close\_ts  }\SpecialCharTok{|\textgreater{}}\NormalTok{ tsibble}\SpecialCharTok{::}\FunctionTok{interval}\NormalTok{()}
\NormalTok{demand\_interval }\OtherTok{\textless{}{-}}\NormalTok{ demand\_ts }\SpecialCharTok{|\textgreater{}}\NormalTok{ tsibble}\SpecialCharTok{::}\FunctionTok{interval}\NormalTok{()}

\NormalTok{bricks\_interval}
\end{Highlighting}
\end{Shaded}

\begin{verbatim}
## <interval[1]>
## [1] 1Q
\end{verbatim}

\begin{Shaded}
\begin{Highlighting}[]
\NormalTok{lynx\_interval}
\end{Highlighting}
\end{Shaded}

\begin{verbatim}
## <interval[1]>
## [1] 1Y
\end{verbatim}

\begin{Shaded}
\begin{Highlighting}[]
\NormalTok{close\_interval}
\end{Highlighting}
\end{Shaded}

\begin{verbatim}
## <interval[1]>
## [1] !
\end{verbatim}

\begin{Shaded}
\begin{Highlighting}[]
\NormalTok{demand\_interval}
\end{Highlighting}
\end{Shaded}

\begin{verbatim}
## <interval[1]>
## [1] 30m
\end{verbatim}

\textbf{Answer (interpretation):}

\begin{itemize}
\tightlist
\item
  \texttt{aus\_production} (\textbf{Bricks}) is \textbf{quarterly}
  (index is \texttt{yearquarter}).
\item
  \texttt{pelt} (\textbf{Lynx}) is \textbf{annual} (index is
  \texttt{year}).
\item
  \texttt{gafa\_stock} (\textbf{Close}) is \textbf{daily trading days}
  (dates exist only on trading days, so the calendar is not strictly
  regular daily).
\item
  \texttt{vic\_elec} (\textbf{Demand}) is \textbf{half-hourly}
  (sub-daily regular interval).
\end{itemize}

\subsection{Time plots (autoplot)}\label{time-plots-autoplot}

\begin{Shaded}
\begin{Highlighting}[]
\CommentTok{\# Bricks}
\NormalTok{bricks\_ts }\SpecialCharTok{|\textgreater{}}
  \FunctionTok{autoplot}\NormalTok{(Bricks) }\SpecialCharTok{+}
  \FunctionTok{labs}\NormalTok{(}\AttributeTok{title =} \StringTok{"Australian brick production"}\NormalTok{, }\AttributeTok{x =} \StringTok{"Quarter"}\NormalTok{, }\AttributeTok{y =} \StringTok{"Bricks"}\NormalTok{)}
\end{Highlighting}
\end{Shaded}

\pandocbounded{\includegraphics[keepaspectratio]{homework1_files/figure-latex/time-plots-autoplot-1.pdf}}

\begin{Shaded}
\begin{Highlighting}[]
\CommentTok{\# Lynx}
\NormalTok{lynx\_ts }\SpecialCharTok{|\textgreater{}}
  \FunctionTok{autoplot}\NormalTok{(value) }\SpecialCharTok{+}
  \FunctionTok{labs}\NormalTok{(}\AttributeTok{title =} \StringTok{"Lynx pelts"}\NormalTok{, }\AttributeTok{x =} \StringTok{"Year"}\NormalTok{, }\AttributeTok{y =} \StringTok{"Pelts"}\NormalTok{)}
\end{Highlighting}
\end{Shaded}

\pandocbounded{\includegraphics[keepaspectratio]{homework1_files/figure-latex/time-plots-autoplot-2.pdf}}

\begin{Shaded}
\begin{Highlighting}[]
\CommentTok{\# Close (all four stocks)}
\NormalTok{gafa\_stock }\SpecialCharTok{|\textgreater{}}
  \FunctionTok{autoplot}\NormalTok{(Close) }\SpecialCharTok{+}
  \FunctionTok{labs}\NormalTok{(}\AttributeTok{title =} \StringTok{"Daily closing prices: GAFA"}\NormalTok{, }\AttributeTok{x =} \StringTok{"Date"}\NormalTok{, }\AttributeTok{y =} \StringTok{"Close"}\NormalTok{)}
\end{Highlighting}
\end{Shaded}

\pandocbounded{\includegraphics[keepaspectratio]{homework1_files/figure-latex/time-plots-autoplot-3.pdf}}

\subsection{Demand plot with modified axis labels and
title}\label{demand-plot-with-modified-axis-labels-and-title}

\begin{Shaded}
\begin{Highlighting}[]
\NormalTok{demand\_ts }\SpecialCharTok{|\textgreater{}}
  \FunctionTok{autoplot}\NormalTok{(Demand) }\SpecialCharTok{+}
  \FunctionTok{labs}\NormalTok{(}
    \AttributeTok{title =} \StringTok{"Victoria electricity demand (half{-}hourly)"}\NormalTok{,}
    \AttributeTok{x =} \StringTok{"Time"}\NormalTok{,}
    \AttributeTok{y =} \StringTok{"Demand (MW)"}
\NormalTok{  )}
\end{Highlighting}
\end{Shaded}

\pandocbounded{\includegraphics[keepaspectratio]{homework1_files/figure-latex/demand-plot-1.pdf}}

\section{Peak closing price day(s) for each GAFA
stock}\label{peak-closing-price-days-for-each-gafa-stock}

Find the trading day(s) (ties possible) corresponding to the maximum
closing price for each of the four stocks.

\begin{Shaded}
\begin{Highlighting}[]
\NormalTok{gafa\_peaks }\OtherTok{\textless{}{-}}\NormalTok{ gafa\_stock }\SpecialCharTok{|\textgreater{}}
  \FunctionTok{group\_by}\NormalTok{(Symbol) }\SpecialCharTok{|\textgreater{}}
  \FunctionTok{filter}\NormalTok{(Close }\SpecialCharTok{==} \FunctionTok{max}\NormalTok{(Close, }\AttributeTok{na.rm =} \ConstantTok{TRUE}\NormalTok{)) }\SpecialCharTok{|\textgreater{}}
  \FunctionTok{arrange}\NormalTok{(Symbol, Date) }\SpecialCharTok{|\textgreater{}}
  \FunctionTok{select}\NormalTok{(Symbol, Date, Close)}

\NormalTok{gafa\_peaks}
\end{Highlighting}
\end{Shaded}

\begin{verbatim}
## # A tsibble: 4 x 3 [!]
## # Key:       Symbol [4]
## # Groups:    Symbol [4]
##   Symbol Date       Close
##   <chr>  <date>     <dbl>
## 1 AAPL   2018-10-03  232.
## 2 AMZN   2018-09-04 2040.
## 3 FB     2018-07-25  218.
## 4 GOOG   2018-07-26 1268.
\end{verbatim}

\textbf{Answer:} The table above lists the peak closing date(s) and peak
\texttt{Close} for each symbol.

\section{tute1.csv: import, tsibble conversion, plots, and
faceting}\label{tute1.csv-import-tsibble-conversion-plots-and-faceting}

\texttt{tute1.csv} contains quarterly series \textbf{Sales},
\textbf{AdBudget}, and \textbf{GDP} (inflation-adjusted).

\subsection{Download and read the
data}\label{download-and-read-the-data}

\begin{Shaded}
\begin{Highlighting}[]
\NormalTok{tute1\_url }\OtherTok{\textless{}{-}} \StringTok{"https://otexts.com/fpp3/extrafiles/tute1.csv"}
\NormalTok{tute1 }\OtherTok{\textless{}{-}}\NormalTok{ readr}\SpecialCharTok{::}\FunctionTok{read\_csv}\NormalTok{(tute1\_url)}

\NormalTok{dplyr}\SpecialCharTok{::}\FunctionTok{glimpse}\NormalTok{(tute1)}
\end{Highlighting}
\end{Shaded}

\begin{verbatim}
## Rows: 100
## Columns: 4
## $ Quarter  <date> 1981-03-01, 1981-06-01, 1981-09-01, 1981-12-01, 1982-03-01, ~
## $ Sales    <dbl> 1020.2, 889.2, 795.0, 1003.9, 1057.7, 944.4, 778.5, 932.5, 99~
## $ AdBudget <dbl> 659.2, 589.0, 512.5, 614.1, 647.2, 602.0, 530.7, 608.4, 637.9~
## $ GDP      <dbl> 251.8, 290.9, 290.8, 292.4, 279.1, 254.0, 295.6, 271.7, 259.6~
\end{verbatim}

\subsection{Convert to a quarterly
tsibble}\label{convert-to-a-quarterly-tsibble}

\begin{Shaded}
\begin{Highlighting}[]
\NormalTok{mytimeseries }\OtherTok{\textless{}{-}}\NormalTok{ tute1 }\SpecialCharTok{|\textgreater{}}
  \FunctionTok{mutate}\NormalTok{(}\AttributeTok{Quarter =} \FunctionTok{yearquarter}\NormalTok{(Quarter)) }\SpecialCharTok{|\textgreater{}}
  \FunctionTok{as\_tsibble}\NormalTok{(}\AttributeTok{index =}\NormalTok{ Quarter)}

\NormalTok{mytimeseries}
\end{Highlighting}
\end{Shaded}

\begin{verbatim}
## # A tsibble: 100 x 4 [1Q]
##    Quarter Sales AdBudget   GDP
##      <qtr> <dbl>    <dbl> <dbl>
##  1 1981 Q1 1020.     659.  252.
##  2 1981 Q2  889.     589   291.
##  3 1981 Q3  795      512.  291.
##  4 1981 Q4 1004.     614.  292.
##  5 1982 Q1 1058.     647.  279.
##  6 1982 Q2  944.     602   254 
##  7 1982 Q3  778.     531.  296.
##  8 1982 Q4  932.     608.  272.
##  9 1983 Q1  996.     638.  260.
## 10 1983 Q2  908.     582.  280.
## # i 90 more rows
\end{verbatim}

\subsection{Plot all three series with
facets}\label{plot-all-three-series-with-facets}

\begin{Shaded}
\begin{Highlighting}[]
\NormalTok{mytimeseries }\SpecialCharTok{|\textgreater{}}
  \FunctionTok{pivot\_longer}\NormalTok{(}\SpecialCharTok{{-}}\NormalTok{Quarter) }\SpecialCharTok{|\textgreater{}}
  \FunctionTok{ggplot}\NormalTok{(}\FunctionTok{aes}\NormalTok{(}\AttributeTok{x =}\NormalTok{ Quarter, }\AttributeTok{y =}\NormalTok{ value, }\AttributeTok{colour =}\NormalTok{ name)) }\SpecialCharTok{+}
  \FunctionTok{geom\_line}\NormalTok{() }\SpecialCharTok{+}
  \FunctionTok{facet\_grid}\NormalTok{(name }\SpecialCharTok{\textasciitilde{}}\NormalTok{ ., }\AttributeTok{scales =} \StringTok{"free\_y"}\NormalTok{) }\SpecialCharTok{+}
  \FunctionTok{labs}\NormalTok{(}\AttributeTok{title =} \StringTok{"Sales, AdBudget, and GDP (facetted)"}\NormalTok{, }\AttributeTok{x =} \StringTok{"Quarter"}\NormalTok{, }\AttributeTok{y =} \StringTok{""}\NormalTok{)}
\end{Highlighting}
\end{Shaded}

\pandocbounded{\includegraphics[keepaspectratio]{homework1_files/figure-latex/three-with-facets-1.pdf}}

\subsection{\texorpdfstring{Plot without
\texttt{facet\_grid()}}{Plot without facet\_grid()}}\label{plot-without-facet_grid}

\begin{Shaded}
\begin{Highlighting}[]
\NormalTok{mytimeseries }\SpecialCharTok{|\textgreater{}}
  \FunctionTok{pivot\_longer}\NormalTok{(}\SpecialCharTok{{-}}\NormalTok{Quarter) }\SpecialCharTok{|\textgreater{}}
  \FunctionTok{ggplot}\NormalTok{(}\FunctionTok{aes}\NormalTok{(}\AttributeTok{x =}\NormalTok{ Quarter, }\AttributeTok{y =}\NormalTok{ value, }\AttributeTok{colour =}\NormalTok{ name)) }\SpecialCharTok{+}
  \FunctionTok{geom\_line}\NormalTok{() }\SpecialCharTok{+}
  \FunctionTok{labs}\NormalTok{(}\AttributeTok{title =} \StringTok{"Sales, AdBudget, and GDP (no faceting)"}\NormalTok{, }\AttributeTok{x =} \StringTok{"Quarter"}\NormalTok{, }\AttributeTok{y =} \StringTok{""}\NormalTok{)}
\end{Highlighting}
\end{Shaded}

\pandocbounded{\includegraphics[keepaspectratio]{homework1_files/figure-latex/without-facetgrid-1.pdf}}

\textbf{Answer:} Without faceting (and a shared y-axis), the
largest-scale series visually dominates and the smaller-scale series
become difficult to compare.

\section{USgas: annual natural gas consumption by state (New
England)}\label{usgas-annual-natural-gas-consumption-by-state-new-england}

\subsection{\texorpdfstring{Create a tsibble from \texttt{us\_total}
(year index, state
key)}{Create a tsibble from us\_total (year index, state key)}}\label{create-a-tsibble-from-us_total-year-index-state-key}

\begin{Shaded}
\begin{Highlighting}[]
\FunctionTok{data}\NormalTok{(}\StringTok{"us\_total"}\NormalTok{, }\AttributeTok{package =} \StringTok{"USgas"}\NormalTok{)}

\NormalTok{gas\_ts }\OtherTok{\textless{}{-}}\NormalTok{ us\_total }\SpecialCharTok{|\textgreater{}}
  \FunctionTok{as\_tsibble}\NormalTok{(}\AttributeTok{index =}\NormalTok{ year, }\AttributeTok{key =}\NormalTok{ state)}

\NormalTok{gas\_ts}
\end{Highlighting}
\end{Shaded}

\begin{verbatim}
## # A tsibble: 1,266 x 3 [1Y]
## # Key:       state [53]
##     year state        y
##    <int> <chr>    <int>
##  1  1997 Alabama 324158
##  2  1998 Alabama 329134
##  3  1999 Alabama 337270
##  4  2000 Alabama 353614
##  5  2001 Alabama 332693
##  6  2002 Alabama 379343
##  7  2003 Alabama 350345
##  8  2004 Alabama 382367
##  9  2005 Alabama 353156
## 10  2006 Alabama 391093
## # i 1,256 more rows
\end{verbatim}

\subsection{Plot New England states}\label{plot-new-england-states}

New England = Maine, Vermont, New Hampshire, Massachusetts, Connecticut,
Rhode Island.

\begin{Shaded}
\begin{Highlighting}[]
\NormalTok{new\_england }\OtherTok{\textless{}{-}} \FunctionTok{c}\NormalTok{(}\StringTok{"Maine"}\NormalTok{, }\StringTok{"Vermont"}\NormalTok{, }\StringTok{"New Hampshire"}\NormalTok{,}
                 \StringTok{"Massachusetts"}\NormalTok{, }\StringTok{"Connecticut"}\NormalTok{, }\StringTok{"Rhode Island"}\NormalTok{)}

\NormalTok{gas\_ts }\SpecialCharTok{|\textgreater{}}
  \FunctionTok{filter}\NormalTok{(state }\SpecialCharTok{\%in\%}\NormalTok{ new\_england) }\SpecialCharTok{|\textgreater{}}
  \FunctionTok{autoplot}\NormalTok{(y) }\SpecialCharTok{+}
  \FunctionTok{labs}\NormalTok{(}
    \AttributeTok{title =} \StringTok{"Annual natural gas consumption (New England states)"}\NormalTok{,}
    \AttributeTok{x =} \StringTok{"Year"}\NormalTok{,}
    \AttributeTok{y =} \StringTok{"Consumption"}
\NormalTok{  )}
\end{Highlighting}
\end{Shaded}

\pandocbounded{\includegraphics[keepaspectratio]{homework1_files/figure-latex/plot-ne-states-1.pdf}}

\section{\texorpdfstring{tourism.xlsx: recreate \texttt{tourism} tsibble
and
summaries}{tourism.xlsx: recreate tourism tsibble and summaries}}\label{tourism.xlsx-recreate-tourism-tsibble-and-summaries}

\subsection{Download and read
tourism.xlsx}\label{download-and-read-tourism.xlsx}

\begin{Shaded}
\begin{Highlighting}[]
\NormalTok{tourism\_url }\OtherTok{\textless{}{-}} \StringTok{"https://otexts.com/fpp3/extrafiles/tourism.xlsx"}
\NormalTok{tourism\_file }\OtherTok{\textless{}{-}} \FunctionTok{tempfile}\NormalTok{(}\AttributeTok{fileext =} \StringTok{".xlsx"}\NormalTok{)}

\FunctionTok{download.file}\NormalTok{(tourism\_url, tourism\_file, }\AttributeTok{mode =} \StringTok{"wb"}\NormalTok{)}

\NormalTok{tourism\_raw }\OtherTok{\textless{}{-}}\NormalTok{ readxl}\SpecialCharTok{::}\FunctionTok{read\_excel}\NormalTok{(tourism\_file)}

\NormalTok{dplyr}\SpecialCharTok{::}\FunctionTok{glimpse}\NormalTok{(tourism\_raw)}
\end{Highlighting}
\end{Shaded}

\begin{verbatim}
## Rows: 24,320
## Columns: 5
## $ Quarter <chr> "1998-01-01", "1998-04-01", "1998-07-01", "1998-10-01", "1999-~
## $ Region  <chr> "Adelaide", "Adelaide", "Adelaide", "Adelaide", "Adelaide", "A~
## $ State   <chr> "South Australia", "South Australia", "South Australia", "Sout~
## $ Purpose <chr> "Business", "Business", "Business", "Business", "Business", "B~
## $ Trips   <dbl> 135.0777, 109.9873, 166.0347, 127.1605, 137.4485, 199.9126, 16~
\end{verbatim}

\subsection{\texorpdfstring{Create a tsibble identical in structure to
\texttt{tsibble::tourism}}{Create a tsibble identical in structure to tsibble::tourism}}\label{create-a-tsibble-identical-in-structure-to-tsibbletourism}

\begin{Shaded}
\begin{Highlighting}[]
\NormalTok{tourism\_ts }\OtherTok{\textless{}{-}}\NormalTok{ tourism\_raw }\SpecialCharTok{|\textgreater{}}
  \FunctionTok{mutate}\NormalTok{(}\AttributeTok{Quarter =} \FunctionTok{yearquarter}\NormalTok{(Quarter)) }\SpecialCharTok{|\textgreater{}}
  \FunctionTok{as\_tsibble}\NormalTok{(}
    \AttributeTok{index =}\NormalTok{ Quarter,}
    \AttributeTok{key =} \FunctionTok{c}\NormalTok{(State, Region, Purpose)}
\NormalTok{  )}

\NormalTok{tourism\_ts}
\end{Highlighting}
\end{Shaded}

\begin{verbatim}
## # A tsibble: 24,320 x 5 [1Q]
## # Key:       State, Region, Purpose [304]
##    Quarter Region   State Purpose  Trips
##      <qtr> <chr>    <chr> <chr>    <dbl>
##  1 1998 Q1 Canberra ACT   Business 150. 
##  2 1998 Q2 Canberra ACT   Business  99.9
##  3 1998 Q3 Canberra ACT   Business 130. 
##  4 1998 Q4 Canberra ACT   Business 102. 
##  5 1999 Q1 Canberra ACT   Business  95.5
##  6 1999 Q2 Canberra ACT   Business 229. 
##  7 1999 Q3 Canberra ACT   Business 109. 
##  8 1999 Q4 Canberra ACT   Business 159. 
##  9 2000 Q1 Canberra ACT   Business 105. 
## 10 2000 Q2 Canberra ACT   Business 202. 
## # i 24,310 more rows
\end{verbatim}

\subsection{Ensure temporal ordering for each
Region/Purpose}\label{ensure-temporal-ordering-for-each-regionpurpose}

\begin{Shaded}
\begin{Highlighting}[]
\NormalTok{tourism\_ts }\OtherTok{\textless{}{-}}\NormalTok{ tourism\_ts }\SpecialCharTok{|\textgreater{}}
  \FunctionTok{arrange}\NormalTok{(Region, Purpose, Quarter)}
\end{Highlighting}
\end{Shaded}

\subsection{Region--Purpose combination with maximum average overnight
trips}\label{regionpurpose-combination-with-maximum-average-overnight-trips}

\begin{Shaded}
\begin{Highlighting}[]
\NormalTok{max\_region\_purpose }\OtherTok{\textless{}{-}}\NormalTok{ tourism\_ts }\SpecialCharTok{|\textgreater{}}
  \FunctionTok{group\_by}\NormalTok{(Region, Purpose) }\SpecialCharTok{|\textgreater{}}
  \FunctionTok{summarise}\NormalTok{(}\AttributeTok{avg\_trips =} \FunctionTok{mean}\NormalTok{(Trips, }\AttributeTok{na.rm =} \ConstantTok{TRUE}\NormalTok{), }\AttributeTok{.groups =} \StringTok{"drop"}\NormalTok{) }\SpecialCharTok{|\textgreater{}}
  \FunctionTok{arrange}\NormalTok{(}\FunctionTok{desc}\NormalTok{(avg\_trips)) }\SpecialCharTok{|\textgreater{}}
  \FunctionTok{slice}\NormalTok{(}\DecValTok{1}\NormalTok{)}

\NormalTok{max\_region\_purpose}
\end{Highlighting}
\end{Shaded}

\begin{verbatim}
## # A tsibble: 1 x 4 [1Q]
## # Key:       Region, Purpose [1]
##   Region    Purpose  Quarter avg_trips
##   <chr>     <chr>      <qtr>     <dbl>
## 1 Melbourne Visiting 2017 Q4      985.
\end{verbatim}

\textbf{Answer:} The row above identifies the
\texttt{(Region,\ Purpose)} combination with the greatest mean
\texttt{Trips}.

\subsection{Total trips by State (combining Region and
Purpose)}\label{total-trips-by-state-combining-region-and-purpose}

\begin{Shaded}
\begin{Highlighting}[]
\NormalTok{tourism\_state\_total }\OtherTok{\textless{}{-}}\NormalTok{ tourism\_ts }\SpecialCharTok{|\textgreater{}}
  \FunctionTok{index\_by}\NormalTok{(Quarter) }\SpecialCharTok{|\textgreater{}}
  \FunctionTok{group\_by}\NormalTok{(State) }\SpecialCharTok{|\textgreater{}}
  \FunctionTok{summarise}\NormalTok{(}\AttributeTok{Trips =} \FunctionTok{sum}\NormalTok{(Trips, }\AttributeTok{na.rm =} \ConstantTok{TRUE}\NormalTok{)) }\SpecialCharTok{|\textgreater{}}
  \FunctionTok{as\_tsibble}\NormalTok{(}\AttributeTok{index =}\NormalTok{ Quarter, }\AttributeTok{key =}\NormalTok{ State)}

\NormalTok{tourism\_state\_total}
\end{Highlighting}
\end{Shaded}

\begin{verbatim}
## # A tsibble: 640 x 3 [1Q]
## # Key:       State [8]
##    State Quarter Trips
##    <chr>   <qtr> <dbl>
##  1 ACT   1998 Q1  551.
##  2 ACT   1998 Q2  416.
##  3 ACT   1998 Q3  436.
##  4 ACT   1998 Q4  450.
##  5 ACT   1999 Q1  379.
##  6 ACT   1999 Q2  558.
##  7 ACT   1999 Q3  449.
##  8 ACT   1999 Q4  595.
##  9 ACT   2000 Q1  600.
## 10 ACT   2000 Q2  557.
## # i 630 more rows
\end{verbatim}

\begin{Shaded}
\begin{Highlighting}[]
\NormalTok{tourism\_state\_total }\SpecialCharTok{|\textgreater{}}
  \FunctionTok{autoplot}\NormalTok{(Trips) }\SpecialCharTok{+}
  \FunctionTok{labs}\NormalTok{(}\AttributeTok{title =} \StringTok{"Total overnight trips by State"}\NormalTok{, }\AttributeTok{x =} \StringTok{"Quarter"}\NormalTok{, }\AttributeTok{y =} \StringTok{"Trips"}\NormalTok{)}
\end{Highlighting}
\end{Shaded}

\pandocbounded{\includegraphics[keepaspectratio]{homework1_files/figure-latex/tourism-state-total-1.pdf}}

\section{aus\_arrivals: compare arrivals from Japan, NZ, UK,
US}\label{aus_arrivals-compare-arrivals-from-japan-nz-uk-us}

\textcolor{red}{(JUST FOR FUN)}

Use \texttt{autoplot()}, \texttt{gg\_season()}, and
\texttt{gg\_subseries()}.

\begin{Shaded}
\begin{Highlighting}[]
\NormalTok{aus\_arrivals }\SpecialCharTok{|\textgreater{}}
  \FunctionTok{autoplot}\NormalTok{(Arrivals) }\SpecialCharTok{+}
  \FunctionTok{labs}\NormalTok{(}\AttributeTok{title =} \StringTok{"International arrivals to Australia"}\NormalTok{, }\AttributeTok{x =} \StringTok{"Quarter"}\NormalTok{, }\AttributeTok{y =} \StringTok{"Arrivals"}\NormalTok{)}
\end{Highlighting}
\end{Shaded}

\pandocbounded{\includegraphics[keepaspectratio]{homework1_files/figure-latex/aus-arrivals-1.pdf}}

\begin{Shaded}
\begin{Highlighting}[]
\NormalTok{aus\_arrivals }\SpecialCharTok{|\textgreater{}}
  \FunctionTok{gg\_season}\NormalTok{(Arrivals) }\SpecialCharTok{+}
  \FunctionTok{labs}\NormalTok{(}\AttributeTok{title =} \StringTok{"Seasonal plot: arrivals to Australia"}\NormalTok{, }\AttributeTok{y =} \StringTok{"Arrivals"}\NormalTok{)}
\end{Highlighting}
\end{Shaded}

\pandocbounded{\includegraphics[keepaspectratio]{homework1_files/figure-latex/aus-arrivals-2.pdf}}

\begin{Shaded}
\begin{Highlighting}[]
\NormalTok{aus\_arrivals }\SpecialCharTok{|\textgreater{}}
  \FunctionTok{gg\_subseries}\NormalTok{(Arrivals) }\SpecialCharTok{+}
  \FunctionTok{labs}\NormalTok{(}\AttributeTok{title =} \StringTok{"Subseries plot: arrivals to Australia"}\NormalTok{, }\AttributeTok{y =} \StringTok{"Arrivals"}\NormalTok{)}
\end{Highlighting}
\end{Shaded}

\pandocbounded{\includegraphics[keepaspectratio]{homework1_files/figure-latex/aus-arrivals-3.pdf}}

\textbf{Answer (unusual observations):}\\
The subseries plot highlights deviations from the typical seasonal
pattern by showing each quarter separately over time. Japan exhibits
unusually low arrivals in Q2 and Q3 during later years, deviating
sharply from its normally strong seasonal structure. The United States
shows occasional Q3 spikes that exceed the typical seasonal range,
indicating one-off surges rather than regular seasonality. The UK also
displays increased variability in Q3 and Q4, with some years standing
out as unusually high. In contrast, New Zealand arrivals remain
relatively stable across all quarters, with no clear unusual
observations.

\section{aus\_retail: sample a series and
explore}\label{aus_retail-sample-a-series-and-explore}

\textcolor{red}{(JUST FOR FUN)}

\begin{Shaded}
\begin{Highlighting}[]
\FunctionTok{set.seed}\NormalTok{(}\DecValTok{67}\NormalTok{)}
\NormalTok{myseries }\OtherTok{\textless{}{-}}\NormalTok{ aus\_retail }\SpecialCharTok{|\textgreater{}}
  \FunctionTok{filter}\NormalTok{(}\StringTok{\textasciigrave{}}\AttributeTok{Series ID}\StringTok{\textasciigrave{}} \SpecialCharTok{==} \FunctionTok{sample}\NormalTok{(aus\_retail}\SpecialCharTok{$}\StringTok{\textasciigrave{}}\AttributeTok{Series ID}\StringTok{\textasciigrave{}}\NormalTok{, }\DecValTok{1}\NormalTok{))}

\NormalTok{myseries }\SpecialCharTok{|\textgreater{}}\NormalTok{ dplyr}\SpecialCharTok{::}\FunctionTok{glimpse}\NormalTok{()}
\end{Highlighting}
\end{Shaded}

\begin{verbatim}
## Rows: 441
## Columns: 5
## Key: State, Industry [1]
## $ State       <chr> "New South Wales", "New South Wales", "New South Wales", "~
## $ Industry    <chr> "Department stores", "Department stores", "Department stor~
## $ `Series ID` <chr> "A3349790V", "A3349790V", "A3349790V", "A3349790V", "A3349~
## $ Month       <mth> 1982 Apr, 1982 May, 1982 Jun, 1982 Jul, 1982 Aug, 1982 Sep~
## $ Turnover    <dbl> 178.3, 202.8, 176.3, 172.6, 169.6, 181.4, 173.9, 206.6, 34~
\end{verbatim}

\subsection{Graphics exploration}\label{graphics-exploration}

\begin{Shaded}
\begin{Highlighting}[]
\NormalTok{myseries }\SpecialCharTok{|\textgreater{}}
  \FunctionTok{autoplot}\NormalTok{(Turnover) }\SpecialCharTok{+}
  \FunctionTok{labs}\NormalTok{(}\AttributeTok{title =} \StringTok{"Selected retail turnover series"}\NormalTok{, }\AttributeTok{x =} \StringTok{"Month"}\NormalTok{, }\AttributeTok{y =} \StringTok{"Turnover"}\NormalTok{)}
\end{Highlighting}
\end{Shaded}

\pandocbounded{\includegraphics[keepaspectratio]{homework1_files/figure-latex/graphics-exploration-1.pdf}}

\begin{Shaded}
\begin{Highlighting}[]
\NormalTok{myseries }\SpecialCharTok{|\textgreater{}}
  \FunctionTok{gg\_season}\NormalTok{(Turnover) }\SpecialCharTok{+}
  \FunctionTok{labs}\NormalTok{(}\AttributeTok{title =} \StringTok{"Seasonal plot: selected retail series"}\NormalTok{, }\AttributeTok{y =} \StringTok{"Turnover"}\NormalTok{)}
\end{Highlighting}
\end{Shaded}

\pandocbounded{\includegraphics[keepaspectratio]{homework1_files/figure-latex/graphics-exploration-2.pdf}}

\begin{Shaded}
\begin{Highlighting}[]
\NormalTok{myseries }\SpecialCharTok{|\textgreater{}}
  \FunctionTok{gg\_subseries}\NormalTok{(Turnover) }\SpecialCharTok{+}
  \FunctionTok{labs}\NormalTok{(}\AttributeTok{title =} \StringTok{"Subseries plot: selected retail series"}\NormalTok{, }\AttributeTok{y =} \StringTok{"Turnover"}\NormalTok{)}
\end{Highlighting}
\end{Shaded}

\pandocbounded{\includegraphics[keepaspectratio]{homework1_files/figure-latex/graphics-exploration-3.pdf}}

\begin{Shaded}
\begin{Highlighting}[]
\NormalTok{myseries }\SpecialCharTok{|\textgreater{}}
  \FunctionTok{gg\_lag}\NormalTok{(Turnover) }\SpecialCharTok{+}
  \FunctionTok{labs}\NormalTok{(}\AttributeTok{title =} \StringTok{"Lag plot: selected retail series"}\NormalTok{, }\AttributeTok{y =} \StringTok{"Turnover"}\NormalTok{)}
\end{Highlighting}
\end{Shaded}

\pandocbounded{\includegraphics[keepaspectratio]{homework1_files/figure-latex/graphics-exploration-4.pdf}}

\begin{Shaded}
\begin{Highlighting}[]
\NormalTok{myseries }\SpecialCharTok{|\textgreater{}}
  \FunctionTok{ACF}\NormalTok{(Turnover) }\SpecialCharTok{|\textgreater{}}
  \FunctionTok{autoplot}\NormalTok{() }\SpecialCharTok{+}
  \FunctionTok{labs}\NormalTok{(}\AttributeTok{title =} \StringTok{"ACF: selected retail series"}\NormalTok{)}
\end{Highlighting}
\end{Shaded}

\pandocbounded{\includegraphics[keepaspectratio]{homework1_files/figure-latex/graphics-exploration-5.pdf}}

\textbf{Answer (seasonality/cyclicity/trend):}\\
- \textbf{Seasonality} is indicated by repeating within-year patterns
and ACF spikes at seasonal lags (e.g., 12 for monthly data).\\
- \textbf{Trend} appears as persistent long-run increase/decrease and
often as slow ACF decay.\\
- \textbf{Cyclicity} shows as multi-year rises/falls not tied to a fixed
seasonal frequency.\\
Use the plots above to describe what is present in \emph{this specific
sampled series}.

\section{Explore five additional series with common graphics +
ACF}\label{explore-five-additional-series-with-common-graphics-acf}

Series: - \textbf{``Total Private'' Employed} from
\texttt{us\_employment} - \textbf{Bricks} from \texttt{aus\_production}
- \textbf{Hare} from \texttt{pelt} - \textbf{``H02'' Cost} from
\texttt{PBS} - \textbf{Barrels} from \texttt{us\_gasoline}

\begin{Shaded}
\begin{Highlighting}[]
\NormalTok{emp\_private }\OtherTok{\textless{}{-}}\NormalTok{ us\_employment }\SpecialCharTok{|\textgreater{}}
  \FunctionTok{filter}\NormalTok{(Title }\SpecialCharTok{==} \StringTok{"Total Private"}\NormalTok{) }\SpecialCharTok{|\textgreater{}}
  \FunctionTok{select}\NormalTok{(Month, Employed)}

\NormalTok{bricks\_ts }\OtherTok{\textless{}{-}}\NormalTok{ aus\_production }\SpecialCharTok{|\textgreater{}} \FunctionTok{select}\NormalTok{(Quarter, Bricks)}

\NormalTok{hare\_ts }\OtherTok{\textless{}{-}} \FunctionTok{pelt\_long}\NormalTok{(pelt) }\SpecialCharTok{|\textgreater{}} \FunctionTok{filter}\NormalTok{(Animal }\SpecialCharTok{==} \StringTok{"Hare"}\NormalTok{)}

\NormalTok{pbs\_keys }\OtherTok{\textless{}{-}} \FunctionTok{key\_vars}\NormalTok{(PBS)}

\NormalTok{pbs\_h02 }\OtherTok{\textless{}{-}}\NormalTok{ PBS }\SpecialCharTok{|\textgreater{}}
  \FunctionTok{filter}\NormalTok{(ATC2 }\SpecialCharTok{==} \StringTok{"H02"}\NormalTok{) }\SpecialCharTok{|\textgreater{}}
  \FunctionTok{group\_by}\NormalTok{(}\FunctionTok{across}\NormalTok{(}\FunctionTok{all\_of}\NormalTok{(}\FunctionTok{key\_vars}\NormalTok{(PBS)))) }\SpecialCharTok{|\textgreater{}}
  \FunctionTok{filter}\NormalTok{(dplyr}\SpecialCharTok{::}\FunctionTok{cur\_group\_id}\NormalTok{() }\SpecialCharTok{==} \DecValTok{1}\NormalTok{) }\SpecialCharTok{|\textgreater{}}
  \FunctionTok{ungroup}\NormalTok{() }\SpecialCharTok{|\textgreater{}}
  \FunctionTok{select}\NormalTok{(Month, Cost)}

\NormalTok{gas\_barrels }\OtherTok{\textless{}{-}}\NormalTok{ us\_gasoline }\SpecialCharTok{|\textgreater{}} \FunctionTok{select}\NormalTok{(Week, Barrels)}
\end{Highlighting}
\end{Shaded}

Helper to run the same exploration:

\begin{Shaded}
\begin{Highlighting}[]
\NormalTok{explore\_series }\OtherTok{\textless{}{-}} \ControlFlowTok{function}\NormalTok{(data, value\_col, }\AttributeTok{title\_prefix =} \StringTok{""}\NormalTok{) \{}
  \FunctionTok{list}\NormalTok{(}
    \AttributeTok{time =}\NormalTok{ data }\SpecialCharTok{|\textgreater{}} \FunctionTok{autoplot}\NormalTok{(\{\{ value\_col \}\}) }\SpecialCharTok{+}
      \FunctionTok{labs}\NormalTok{(}\AttributeTok{title =} \FunctionTok{paste0}\NormalTok{(title\_prefix, }\StringTok{" — time plot"}\NormalTok{)),}

    \AttributeTok{season =} \FunctionTok{tryCatch}\NormalTok{(}
\NormalTok{      data }\SpecialCharTok{|\textgreater{}} \FunctionTok{gg\_season}\NormalTok{(\{\{ value\_col \}\}) }\SpecialCharTok{+} \FunctionTok{labs}\NormalTok{(}\AttributeTok{title =} \FunctionTok{paste0}\NormalTok{(title\_prefix, }\StringTok{" — seasonal plot"}\NormalTok{)),}
      \AttributeTok{error =} \ControlFlowTok{function}\NormalTok{(e) }\ConstantTok{NULL}
\NormalTok{    ),}

    \AttributeTok{subseries =} \FunctionTok{tryCatch}\NormalTok{(}
\NormalTok{      data }\SpecialCharTok{|\textgreater{}} \FunctionTok{gg\_subseries}\NormalTok{(\{\{ value\_col \}\}) }\SpecialCharTok{+} \FunctionTok{labs}\NormalTok{(}\AttributeTok{title =} \FunctionTok{paste0}\NormalTok{(title\_prefix, }\StringTok{" — subseries plot"}\NormalTok{)),}
      \AttributeTok{error =} \ControlFlowTok{function}\NormalTok{(e) }\ConstantTok{NULL}
\NormalTok{    ),}

    \AttributeTok{lag =} \FunctionTok{tryCatch}\NormalTok{(}
\NormalTok{      data }\SpecialCharTok{|\textgreater{}} \FunctionTok{gg\_lag}\NormalTok{(\{\{ value\_col \}\}) }\SpecialCharTok{+} \FunctionTok{labs}\NormalTok{(}\AttributeTok{title =} \FunctionTok{paste0}\NormalTok{(title\_prefix, }\StringTok{" — lag plot"}\NormalTok{)),}
      \AttributeTok{error =} \ControlFlowTok{function}\NormalTok{(e) }\ConstantTok{NULL}
\NormalTok{    ),}

    \AttributeTok{acf =}\NormalTok{ data }\SpecialCharTok{|\textgreater{}} \FunctionTok{ACF}\NormalTok{(\{\{ value\_col \}\}) }\SpecialCharTok{|\textgreater{}} \FunctionTok{autoplot}\NormalTok{() }\SpecialCharTok{+}
      \FunctionTok{labs}\NormalTok{(}\AttributeTok{title =} \FunctionTok{paste0}\NormalTok{(title\_prefix, }\StringTok{" — ACF"}\NormalTok{))}
\NormalTok{  )}
\NormalTok{\}}
\end{Highlighting}
\end{Shaded}

\begin{Shaded}
\begin{Highlighting}[]
\NormalTok{plots\_emp }\OtherTok{\textless{}{-}} \FunctionTok{explore\_series}\NormalTok{(emp\_private, Employed, }\StringTok{"US Employment (Total Private)"}\NormalTok{)}
\NormalTok{plots\_emp}\SpecialCharTok{$}\NormalTok{time; plots\_emp}\SpecialCharTok{$}\NormalTok{season; plots\_emp}\SpecialCharTok{$}\NormalTok{subseries; plots\_emp}\SpecialCharTok{$}\NormalTok{lag; plots\_emp}\SpecialCharTok{$}\NormalTok{acf}
\end{Highlighting}
\end{Shaded}

\pandocbounded{\includegraphics[keepaspectratio]{homework1_files/figure-latex/plots-gas-emp-hare-pbs-bricks-1.pdf}}
\pandocbounded{\includegraphics[keepaspectratio]{homework1_files/figure-latex/plots-gas-emp-hare-pbs-bricks-2.pdf}}
\pandocbounded{\includegraphics[keepaspectratio]{homework1_files/figure-latex/plots-gas-emp-hare-pbs-bricks-3.pdf}}
\pandocbounded{\includegraphics[keepaspectratio]{homework1_files/figure-latex/plots-gas-emp-hare-pbs-bricks-4.pdf}}
\pandocbounded{\includegraphics[keepaspectratio]{homework1_files/figure-latex/plots-gas-emp-hare-pbs-bricks-5.pdf}}

\begin{Shaded}
\begin{Highlighting}[]
\NormalTok{plots\_bricks }\OtherTok{\textless{}{-}} \FunctionTok{explore\_series}\NormalTok{(bricks\_ts, Bricks, }\StringTok{"Australian Bricks Production"}\NormalTok{)}
\NormalTok{plots\_bricks}\SpecialCharTok{$}\NormalTok{time; plots\_bricks}\SpecialCharTok{$}\NormalTok{season; plots\_bricks}\SpecialCharTok{$}\NormalTok{subseries; plots\_bricks}\SpecialCharTok{$}\NormalTok{lag; plots\_bricks}\SpecialCharTok{$}\NormalTok{acf}
\end{Highlighting}
\end{Shaded}

\pandocbounded{\includegraphics[keepaspectratio]{homework1_files/figure-latex/plots-gas-emp-hare-pbs-bricks-6.pdf}}
\pandocbounded{\includegraphics[keepaspectratio]{homework1_files/figure-latex/plots-gas-emp-hare-pbs-bricks-7.pdf}}
\pandocbounded{\includegraphics[keepaspectratio]{homework1_files/figure-latex/plots-gas-emp-hare-pbs-bricks-8.pdf}}
\pandocbounded{\includegraphics[keepaspectratio]{homework1_files/figure-latex/plots-gas-emp-hare-pbs-bricks-9.pdf}}
\pandocbounded{\includegraphics[keepaspectratio]{homework1_files/figure-latex/plots-gas-emp-hare-pbs-bricks-10.pdf}}

\begin{Shaded}
\begin{Highlighting}[]
\NormalTok{plots\_hare }\OtherTok{\textless{}{-}} \FunctionTok{explore\_series}\NormalTok{(hare\_ts, value, }\StringTok{"Hare Pelts"}\NormalTok{)}
\NormalTok{plots\_hare}\SpecialCharTok{$}\NormalTok{time; plots\_hare}\SpecialCharTok{$}\NormalTok{season; plots\_hare}\SpecialCharTok{$}\NormalTok{subseries; plots\_hare}\SpecialCharTok{$}\NormalTok{lag; plots\_hare}\SpecialCharTok{$}\NormalTok{acf}
\end{Highlighting}
\end{Shaded}

\pandocbounded{\includegraphics[keepaspectratio]{homework1_files/figure-latex/plots-gas-emp-hare-pbs-bricks-11.pdf}}

\begin{verbatim}
## NULL
\end{verbatim}

\pandocbounded{\includegraphics[keepaspectratio]{homework1_files/figure-latex/plots-gas-emp-hare-pbs-bricks-12.pdf}}
\pandocbounded{\includegraphics[keepaspectratio]{homework1_files/figure-latex/plots-gas-emp-hare-pbs-bricks-13.pdf}}
\pandocbounded{\includegraphics[keepaspectratio]{homework1_files/figure-latex/plots-gas-emp-hare-pbs-bricks-14.pdf}}

\begin{Shaded}
\begin{Highlighting}[]
\NormalTok{plots\_pbs }\OtherTok{\textless{}{-}} \FunctionTok{explore\_series}\NormalTok{(pbs\_h02, Cost, }\StringTok{"PBS H02 Cost"}\NormalTok{)}
\NormalTok{plots\_pbs}\SpecialCharTok{$}\NormalTok{time; plots\_pbs}\SpecialCharTok{$}\NormalTok{season; plots\_pbs}\SpecialCharTok{$}\NormalTok{subseries; plots\_pbs}\SpecialCharTok{$}\NormalTok{lag; plots\_pbs}\SpecialCharTok{$}\NormalTok{acf}
\end{Highlighting}
\end{Shaded}

\pandocbounded{\includegraphics[keepaspectratio]{homework1_files/figure-latex/plots-gas-emp-hare-pbs-bricks-15.pdf}}
\pandocbounded{\includegraphics[keepaspectratio]{homework1_files/figure-latex/plots-gas-emp-hare-pbs-bricks-16.pdf}}
\pandocbounded{\includegraphics[keepaspectratio]{homework1_files/figure-latex/plots-gas-emp-hare-pbs-bricks-17.pdf}}
\pandocbounded{\includegraphics[keepaspectratio]{homework1_files/figure-latex/plots-gas-emp-hare-pbs-bricks-18.pdf}}
\pandocbounded{\includegraphics[keepaspectratio]{homework1_files/figure-latex/plots-gas-emp-hare-pbs-bricks-19.pdf}}

\begin{Shaded}
\begin{Highlighting}[]
\NormalTok{plots\_gas }\OtherTok{\textless{}{-}} \FunctionTok{explore\_series}\NormalTok{(gas\_barrels, Barrels, }\StringTok{"US Gasoline Barrels"}\NormalTok{)}
\NormalTok{plots\_gas}\SpecialCharTok{$}\NormalTok{time; plots\_gas}\SpecialCharTok{$}\NormalTok{season; plots\_gas}\SpecialCharTok{$}\NormalTok{subseries; plots\_gas}\SpecialCharTok{$}\NormalTok{lag; plots\_gas}\SpecialCharTok{$}\NormalTok{acf}
\end{Highlighting}
\end{Shaded}

\pandocbounded{\includegraphics[keepaspectratio]{homework1_files/figure-latex/plots-gas-emp-hare-pbs-bricks-20.pdf}}
\pandocbounded{\includegraphics[keepaspectratio]{homework1_files/figure-latex/plots-gas-emp-hare-pbs-bricks-21.pdf}}
\pandocbounded{\includegraphics[keepaspectratio]{homework1_files/figure-latex/plots-gas-emp-hare-pbs-bricks-22.pdf}}
\pandocbounded{\includegraphics[keepaspectratio]{homework1_files/figure-latex/plots-gas-emp-hare-pbs-bricks-23.pdf}}
\pandocbounded{\includegraphics[keepaspectratio]{homework1_files/figure-latex/plots-gas-emp-hare-pbs-bricks-24.pdf}}

\textbf{Answer (what we learn):} - \textbf{US Gasoline Barrels} and
\textbf{PBS H02 Cost} show clear \textbf{seasonality}: repeating
within‑year patterns and ACF spikes at seasonal lags (and multiples). -
\textbf{US Employment (Total Private)} shows a strong \textbf{trend /
nonstationarity}: long‑run drift in the time plot and a slowly decaying
ACF. - \textbf{Hare Pelts} shows \textbf{cycles} over multiple years:
oscillations in the time plot with an ACF that alternates signs. -
\textbf{Australian Bricks Production} shows \textbf{unusual
years/structural change}: a sharp drop in the early 1980s followed by a
sustained lower level and higher volatility from the early 1990s onward.

\section{Match time plots to ACF
plots}\label{match-time-plots-to-acf-plots}

\textcolor{red}{(JUST FOR FUN)}

The book's figure is included below for matching.

\begin{Shaded}
\begin{Highlighting}[]
\NormalTok{acf\_url }\OtherTok{\textless{}{-}} \StringTok{"https://otexts.com/fpp3/fpp\_files/figure{-}html/acfguess{-}1.png"}
\NormalTok{acf\_file }\OtherTok{\textless{}{-}} \FunctionTok{file.path}\NormalTok{(}\FunctionTok{tempdir}\NormalTok{(), }\StringTok{"acfguess{-}1.png"}\NormalTok{)}

\FunctionTok{download.file}\NormalTok{(acf\_url, acf\_file, }\AttributeTok{mode =} \StringTok{"wb"}\NormalTok{)}

\NormalTok{knitr}\SpecialCharTok{::}\FunctionTok{include\_graphics}\NormalTok{(acf\_file)}
\end{Highlighting}
\end{Shaded}

\includegraphics[width=7.87in]{../../../../../../private/var/folders/z7/hslcs7d13097n9zmv9b992xr0000gn/T/Rtmp7QNPNw/acfguess-1}

\textbf{Answer:} Based on the figure: - \textbf{Time plot A} matches
\textbf{ACF B} - \textbf{Time plot B} matches \textbf{ACF A} -
\textbf{Time plot C} matches \textbf{ACF D} - \textbf{Time plot D}
matches \textbf{ACF C}

\textbf{Justification (brief):} - \textbf{B → A:} near‑zero,
short‑memory ACF fits the cow temperature series. - \textbf{A → B:}
strong seasonal spikes (esp.~lag 12) fit monthly accidental deaths. -
\textbf{D → C:} high, slowly decaying ACF fits the trending air
passengers series. - \textbf{C → D:} oscillating, alternating‑sign ACF
fits mink trappings cycles.

\section{aus\_livestock pigs in Victoria (1990--1995): autoplot + ACF;
compare to white
noise}\label{aus_livestock-pigs-in-victoria-19901995-autoplot-acf-compare-to-white-noise}

\textcolor{red}{(JUST FOR FUN)}

\begin{Shaded}
\begin{Highlighting}[]
\NormalTok{pigs\_vic\_9095 }\OtherTok{\textless{}{-}}\NormalTok{ aus\_livestock }\SpecialCharTok{|\textgreater{}}
  \FunctionTok{filter}\NormalTok{(Animal }\SpecialCharTok{==} \StringTok{"Pigs"}\NormalTok{, State }\SpecialCharTok{==} \StringTok{"Victoria"}\NormalTok{) }\SpecialCharTok{|\textgreater{}}
  \FunctionTok{filter}\NormalTok{(}\FunctionTok{year}\NormalTok{(Month) }\SpecialCharTok{\textgreater{}=} \DecValTok{1990}\NormalTok{, }\FunctionTok{year}\NormalTok{(Month) }\SpecialCharTok{\textless{}=} \DecValTok{1995}\NormalTok{)}

\NormalTok{pigs\_vic\_9095 }\SpecialCharTok{|\textgreater{}}
  \FunctionTok{autoplot}\NormalTok{(Count) }\SpecialCharTok{+}
  \FunctionTok{labs}\NormalTok{(}\AttributeTok{title =} \StringTok{"Pigs slaughtered in Victoria (1990–1995)"}\NormalTok{, }\AttributeTok{x =} \StringTok{"Month"}\NormalTok{, }\AttributeTok{y =} \StringTok{"Count"}\NormalTok{)}
\end{Highlighting}
\end{Shaded}

\pandocbounded{\includegraphics[keepaspectratio]{homework1_files/figure-latex/aus-livestock-pigs-1.pdf}}

\begin{Shaded}
\begin{Highlighting}[]
\NormalTok{pigs\_vic\_9095 }\SpecialCharTok{|\textgreater{}}
  \FunctionTok{ACF}\NormalTok{(Count) }\SpecialCharTok{|\textgreater{}}
  \FunctionTok{autoplot}\NormalTok{() }\SpecialCharTok{+}
  \FunctionTok{labs}\NormalTok{(}\AttributeTok{title =} \StringTok{"ACF: Pigs slaughtered in Victoria (1990–1995)"}\NormalTok{)}
\end{Highlighting}
\end{Shaded}

\pandocbounded{\includegraphics[keepaspectratio]{homework1_files/figure-latex/aus-livestock-pigs-2.pdf}}

Compare to longer period:

\begin{Shaded}
\begin{Highlighting}[]
\NormalTok{pigs\_vic\_full }\OtherTok{\textless{}{-}}\NormalTok{ aus\_livestock }\SpecialCharTok{|\textgreater{}}
  \FunctionTok{filter}\NormalTok{(Animal }\SpecialCharTok{==} \StringTok{"Pigs"}\NormalTok{, State }\SpecialCharTok{==} \StringTok{"Victoria"}\NormalTok{)}

\NormalTok{pigs\_vic\_full }\SpecialCharTok{|\textgreater{}}
  \FunctionTok{ACF}\NormalTok{(Count) }\SpecialCharTok{|\textgreater{}}
  \FunctionTok{autoplot}\NormalTok{() }\SpecialCharTok{+}
  \FunctionTok{labs}\NormalTok{(}\AttributeTok{title =} \StringTok{"ACF: Pigs slaughtered in Victoria (Full series)"}\NormalTok{)}
\end{Highlighting}
\end{Shaded}

\pandocbounded{\includegraphics[keepaspectratio]{homework1_files/figure-latex/pigs-longer-period-1.pdf}}

\textbf{Answer:}\\
- Compared with \textbf{white noise}, this series shows
\textbf{structured autocorrelation} (significant ACF spikes rather than
hovering near zero), indicating dependence and often
seasonality/persistence. - Using a \textbf{longer period} typically
yields a \textbf{more stable ACF estimate} and can make seasonal and
low-frequency structure clearer.

\section{Google stock daily changes: re-indexing, differences, ACF;
assess white
noise}\label{google-stock-daily-changes-re-indexing-differences-acf-assess-white-noise}

\textcolor{red}{(JUST FOR FUN)}

Compute daily changes in GOOG closing prices:

\begin{Shaded}
\begin{Highlighting}[]
\NormalTok{dgoog }\OtherTok{\textless{}{-}}\NormalTok{ gafa\_stock }\SpecialCharTok{|\textgreater{}}
  \FunctionTok{filter}\NormalTok{(Symbol }\SpecialCharTok{==} \StringTok{"GOOG"}\NormalTok{, }\FunctionTok{year}\NormalTok{(Date) }\SpecialCharTok{\textgreater{}=} \DecValTok{2018}\NormalTok{) }\SpecialCharTok{|\textgreater{}}
  \FunctionTok{mutate}\NormalTok{(}\AttributeTok{trading\_day =} \FunctionTok{row\_number}\NormalTok{()) }\SpecialCharTok{|\textgreater{}}
  \FunctionTok{update\_tsibble}\NormalTok{(}\AttributeTok{index =}\NormalTok{ trading\_day, }\AttributeTok{regular =} \ConstantTok{TRUE}\NormalTok{) }\SpecialCharTok{|\textgreater{}}
  \FunctionTok{mutate}\NormalTok{(}\AttributeTok{diff =} \FunctionTok{difference}\NormalTok{(Close))}

\NormalTok{dgoog }\SpecialCharTok{|\textgreater{}}\NormalTok{ dplyr}\SpecialCharTok{::}\FunctionTok{glimpse}\NormalTok{()}
\end{Highlighting}
\end{Shaded}

\begin{verbatim}
## Rows: 251
## Columns: 10
## Key: Symbol [1]
## $ Symbol      <chr> "GOOG", "GOOG", "GOOG", "GOOG", "GOOG", "GOOG", "GOOG", "G~
## $ Date        <date> 2018-01-02, 2018-01-03, 2018-01-04, 2018-01-05, 2018-01-0~
## $ Open        <dbl> 1048.34, 1064.31, 1088.00, 1094.00, 1102.23, 1109.40, 1097~
## $ High        <dbl> 1066.940, 1086.290, 1093.570, 1104.250, 1111.270, 1110.570~
## $ Low         <dbl> 1045.230, 1063.210, 1084.002, 1092.000, 1101.620, 1101.231~
## $ Close       <dbl> 1065.00, 1082.48, 1086.40, 1102.23, 1106.94, 1106.26, 1102~
## $ Adj_Close   <dbl> 1065.00, 1082.48, 1086.40, 1102.23, 1106.94, 1106.26, 1102~
## $ Volume      <dbl> 1237600, 1430200, 1004600, 1279100, 1047600, 902500, 10428~
## $ trading_day <int> 1, 2, 3, 4, 5, 6, 7, 8, 9, 10, 11, 12, 13, 14, 15, 16, 17,~
## $ diff        <dbl> NA, 17.479980, 3.920044, 15.829956, 4.709961, -0.679931, -~
\end{verbatim}

\subsection{Why re-index the tsibble?}\label{why-re-index-the-tsibble}

\textbf{Answer:} The \texttt{Date} index is not strictly regular because
trading does not occur on weekends/holidays. Re-indexing by
\texttt{trading\_day} creates a \textbf{regular} index where each step
is ``next observed trading day,'' so lag operations (and the ACF) align
with equal steps.

\subsection{Plot differences and their
ACF}\label{plot-differences-and-their-acf}

\begin{Shaded}
\begin{Highlighting}[]
\NormalTok{dgoog }\SpecialCharTok{|\textgreater{}}
  \FunctionTok{autoplot}\NormalTok{(diff) }\SpecialCharTok{+}
  \FunctionTok{labs}\NormalTok{(}\AttributeTok{title =} \StringTok{"Daily changes in GOOG closing price"}\NormalTok{, }\AttributeTok{x =} \StringTok{"Trading day index"}\NormalTok{, }\AttributeTok{y =} \StringTok{"Difference in Close"}\NormalTok{)}
\end{Highlighting}
\end{Shaded}

\pandocbounded{\includegraphics[keepaspectratio]{homework1_files/figure-latex/diff-and-acf-1.pdf}}

\begin{Shaded}
\begin{Highlighting}[]
\NormalTok{dgoog }\SpecialCharTok{|\textgreater{}}
  \FunctionTok{ACF}\NormalTok{(diff) }\SpecialCharTok{|\textgreater{}}
  \FunctionTok{autoplot}\NormalTok{() }\SpecialCharTok{+}
  \FunctionTok{labs}\NormalTok{(}\AttributeTok{title =} \StringTok{"ACF: Daily changes in GOOG closing price"}\NormalTok{)}
\end{Highlighting}
\end{Shaded}

\pandocbounded{\includegraphics[keepaspectratio]{homework1_files/figure-latex/diff-and-acf-2.pdf}}

\subsection{Do the changes look like white
noise?}\label{do-the-changes-look-like-white-noise}

\textbf{Answer:} Price \textbf{changes} often show weak linear
autocorrelation (ACF near zero at most lags), consistent with white
noise in the mean. However, volatility clustering (periods of
larger/smaller variability) can still appear in the time plot, so the
series may not be i.i.d. white noise in a strict sense.

\end{document}
